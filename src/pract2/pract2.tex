\section*{\LARGE{Цель практической работы}}
\addcontentsline{toc}{section}{Цель практической работы}
Изучить структуру и функционал рассматриваемой
информационной системы, освоить правила посторения диаграммы вариантов
использования.\par
\textbf{Задачи}
\begin{itemize}
	\item изучить предметную область по заданным вариантам;
	\item определить на концептуальном уровне состав элементов системы;
	\item описать функции рассматриваемой системы с помощью диаграммы вариантов использования.
\end{itemize}
\newpage

\section*{\LARGE{Выполнение практической работы}}
\addcontentsline{toc}{section}{Выполнение практической работы}

\section{Выбор варианта}
Вариант практической работы: 27. Моделирование организации составления расписания спектаклей
кукольного театра.\par

\section{Анализ существующих решений}
Похожие сайты на рынке существуют уже давно, одни из них коммерческие, другие – нет и  направлены на предоставление
той или иной справочной информации\par
Например, сайт Афиша этот сайт предоставляет информацию о спектаклях, которые будут показаны в театре, кино, концертных
залах и т.д. Список и состав мероприятий очень большой. Можно оценить мероприятие по рейтингу, посмотреть отзовы других
людей, посмотреть фотографии с мероприятия. Так же можно посмотреть расписание спектаклей, которые будут показаны в
определенный день и время. Так же можно купить билеты на мероприятие. Но данный сайт не предоставляет информацию о
спецефичных мероприятиях как кукольный театр. Так же данный сайт не предоставляет возможность составить расписание для
кукольного театра только посмотреть расписание на определенный день.\par
1С предоставляет программу для учета кукольного театра и составления расписания спектаклей. Но данная программа
предназначенна для прямой интеграции в бизнес и требует определенных знаний в области учета и подключения к бюджетным
расчетам.\par
Сайт visme предоставляет возможность создавать расписания для различных мероприятий. Но данный сайт не предоставляет
возможность создавать расписание для кукольного театра так как дизайт данного сайта не предназначен для кукольного
театра.

\section{Необходимые функции}
На основе рассмотренных решений были выделены следующие функции:
\begin{itemize}
	\item Создать базу данных для хранения информации о спектаклях и пользователей.
	\item Создасть интерфейс для ввода информации о спектаклях;
	\item Создать интерфейс для ввода информации о пользователях;
	\item Создать интерфейс для составления расписания спектаклей;
	\item Создать интерфейс для просмотра расписания спектаклей;
	\item Создать страницу с информацией о сайте;
	\item Предоставить возможность регистрации пользователей через социальные сети;
	\item Предоставить возможность регистрации пользователей через почту;
	\item Предоставить возможность восстановления пароля;
\end{itemize}

\section{Решения}
\centering
\begin{tabular}{|*{3}{c|}}
	\textbf{ИЛИ} & Истина & Ложь \\[2mm]
	\hline\hline
	Истина & Истина & Истина \\
	Ложь & Истина & Ложь
\end{tabular}


\subsection*{Вывод}
В ходе практической работы было создано первое приложение для Android. Научились создавать проекты и запускать приложение в режиме отладки. Так же узнали об основах разработки приложения для Android, такие как создание разметки с текстовыми полями, полями ввода текста и кнопки, и обработку ввода информации пользователем.
