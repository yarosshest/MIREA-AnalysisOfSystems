\section*{\LARGE{Цель практической работы}}
\addcontentsline{toc}{section}{Цель практической работы}
Изучить структуру и функционал рассматриваемой
информационной системы, освоить правила постарения диаграммы вариантов
использования.\par
\textbf{Задачи}
\begin{itemize}
	\item изучить предметную область по заданным вариантам;
	\item определить на концептуальном уровне состав элементов системы;
	\item описать функции рассматриваемой системы с помощью диаграммы вариантов использования.
\end{itemize}
\newpage

\section*{\LARGE{Выполнение практической работы}}
\addcontentsline{toc}{section}{Выполнение практической работы}

\section{Выбор варианта}
Вариант практической работы: 27. Моделирование организации составления расписания спектаклей
кукольного театра.\par
\newpage

\section{Анализ существующих решений}
Похожие сайты на рынке существуют уже давно, одни из них коммерческие, другие – нет и  направлены на предоставление
той или иной справочной информации\par
Например, сайт Афиша этот сайт предоставляет информацию о спектаклях, которые будут показаны в театре, кино, концертных
залах и т.д. Список и состав мероприятий очень большой.
Можно оценить мероприятие по рейтингу, посмотреть отзывы других
людей, посмотреть фотографии с мероприятия.
Так же можно посмотреть расписание спектаклей, которые будут показаны в
определенный день и время.
Так же можно купить билеты на мероприятие.
Но данный сайт не предоставляет информацию о
специфичных мероприятиях как кукольный театр.
Так же данный сайт не предоставляет возможность составить расписание для
кукольного театра только посмотреть расписание на определенный день.\par
1С предоставляет программу для учета кукольного театра и составления расписания спектаклей.
Но данная программа
предназначенная для прямой интеграции в бизнес и требует определенных знаний в области учета и подключения к бюджетным
расчетам.\par
Сайт visme предоставляет возможность создавать расписания для различных мероприятий.
Но данный сайт не предоставляет
возможность создавать расписание для кукольного театра так как дизайн данного сайта не предназначен для кукольного
театра.
\newpage

\section{Необходимые функции}
На основе рассмотренных решений были выделены следующие функции:
\begin{itemize}
	\item Создать базу данных для хранения информации о спектаклях и пользователей.
	\item Создать интерфейс для ввода информации о спектаклях;
	\item Создать интерфейс для ввода информации о пользователях;
	\item Создать интерфейс для составления расписания спектаклей;
	\item Создать интерфейс для просмотра расписания спектаклей;
	\item Создать страницу с информацией о сайте;
	\item Предоставить возможность регистрации пользователей через социальные сети;
	\item Предоставить возможность регистрации пользователей через почту;
	\item Предоставить возможность восстановления пароля;
\end{itemize}
\newpage

\section{Решения}

\begin{center}
	\begin{tabular}{|l|p{7 cm}|}
		\hline
		\textbf{Наименование} & \textbf{Краткое описание} \\ \hline
		Интерактивное расписание & Создать страницу с календарем, на каждый день которого можно нажать и посмотреть расписание  на тот или иной день \\ \hline
		Регистрация через социальные сети & Предоставить возможность регистрации пользователей через социальные сети \\ \hline
		Регистрация через почту & Предоставить возможность регистрации пользователей через почту \\ \hline
		Восстановление пароля & Предоставить возможность восстановления пароля \\ \hline
		Ввод информации о спектаклях & Создать страницу для ввода информации о спектаклях \\ \hline
		Ввод информации о пользователях & Создать страницу для ввода информации о пользователях \\ \hline
		Составление расписания спектаклей & Создать страницу для составления расписания спектаклей \\ \hline
		Просмотр расписания спектаклей & Создать страницу для просмотра расписания спектаклей \\ \hline
		Просмотр информации о сайте & Создать страницу для просмотра информации о сайте \\ \hline
	\end{tabular}
\end{center}
\newpage

\section{Ожидаемые результаты реализации}
Увеличение количества привлеченных клиентов, увеличение прибыли,уменьшение времени поиска спектакля, уменьшение скорости предоставления нужной информации.
\newpage

\section{Диаграмма вариантов использования}
\img{User_Case.png}{Диаграмма вариантов использования}


\newpage

\subsection*{Вывод}
\addcontentsline{toc}{section}{Вывод}
Выполнен анализ предметной области, изучены
существующие на рынке наиболее известные сайты по расписанию мероприятий.
На основе проведенного анализа выявлены основные «узкие»
места и выявлены возможности для их решения.
Для удобства основные
функции реализуемой системы описаны в виде таблицы.
Создана диаграмма
вариантов использования, определены актеры и основные прецеденты,
установлены связи между ними.
\newpage

\subsection*{Ответы на вопросы}
\addcontentsline{toc}{section}{Ответы на вопросы}
\textbf{Для чего используется язык UML?}\par
Для визуализации, специфицирования, конструирования и документирования артефактов программных систем\par

\textbf{Какие диаграммы входят в состав языка UML?}\par
Классов, Компонентов, Композитной/Составной структуры, Развёртывания, Объектов, Пакетов, Деятельности, Автомата,
Прецедентов, Коммуникации и последовательности, Обзора взаимодействия, Синхронизации\par

\textbf{В чем смысл варианта использования?}\par
Вариант использования описывает, с точки зрения действующего лица, группу действий в системе,
которые приводят к конкретному результату. \par
Варианты использования являются описаниями типичных взаимодействий между пользователями системы и самой системой.
Они отображают внешний интерфейс системы и указывают форму того, что система должна сделать.

\textbf{Каково назначение диаграмм вариантов использования?}\par
Диаграмма прецедентов или диаграмма вариантов использования — диаграмма, на которой отражены отношения,
существующие между акторами и вариантами использования.\par
Основная задача — представлять собой единое средство, дающее возможность заказчику, конечному пользователю и
разработчику совместно обсуждать функциональность и поведение системы.\par

\textbf{Назовите основные свойства вариантов использования.}\par
\begin{itemize}
	\item Вариант использования должен быть независимым от других вариантов использования.
	\item Вариант использования должен быть независимым от реализации.
	\item Вариант использования должен быть независимым от внешнего вида.
	\item Вариант использования должен быть независимым от технологии реализации.
\end{itemize}
\par

\textbf{Назовите основные компоненты диаграмм вариантов использования.}\par
\begin{itemize}
	\item Участник
	\item Варианты использования
	\item Ненаправленная ассоциация
	\item Направленная ассоциация
	\item Обобщение
	\item Зависимость
	\item Точка изгиба связей
	\item Комментарий
	\item Коннектор комментария
\end{itemize}

\textbf{Что такое «действующее лицо»?}\par
Действующее лицо является внешним источником (не элементом системы),
который взаимодействует с системой через вариант использования.
Действующие лица могут быть как реальными людьми (например, пользователями системы),
так и другими компьютерными системами или внешними событиями.\par
\vfill
\textbf{Какую роль могут играть действующие лица по отношению к варианту использования?}\par
Действующие лица представляют не физических людей или системы, а их роли.
Это означает, что когда человек взаимодействует с системой различными способами (предполагая различные роли),
он отображается несколькими действующими лицами.
Например, человек, работающий в службе поддержки и принимающий от клиентов заказы,
будет отображаться в системе как «участник отдела поддержки» и «участник отдела продаж».\par
Действующие лица могут иметь два типа связей с вариантами использования:
Простая ассоциация — отражается линией между актером и вариантом использования (без стрелки).
Отражает связь актера и варианта использования.
простая ассоциация
Направленная ассоциация — то же что и простая ассоциация, но показывает, что вариант использования инициализируется актером.
Обозначается стрелкой.