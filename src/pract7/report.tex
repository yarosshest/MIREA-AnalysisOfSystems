\graphicspath{{./img}} % path to graphics

\section*{\LARGE Цель практической работы}
\addcontentsline{toc}{section}{Цель практической работы}

\textbf{Цель работы} --- сформировать навык построения диаграммы
развертывания в Visual Paradigm.

\textbf{Задачи}:
\begin{itemize}
	\item ознакомиться с назначением диаграммы развертывания и ее
		элементами;
	\item изучить возможности Visual Paradigm для построения
		диаграммы развертывания;
	\item построить диаграмму развертывания в Visual Paradigm в
		соответствии с предложенными заданиями.
\end{itemize}

\clearpage

\section*{\LARGE Выполнение практической работы}
\addcontentsline{toc}{section}{Выполнение практической работы}
\section{Создание диаграммы развертывания}
При создании распределенных или клиент-серверных приложений
требуется визуализировать сетевую инфраструктуру программной системы.\par
Диаграмма развертывания (deployment diagram) предназначена для
представления общей конфигурации или топологии распределения
программной системы и содержит изображение размещения различных
артефактов (исполняемых компонентов и динамических библиотек) по
отдельным узлам системы. Она визуализирует только те элементы физического
представления модели, которые существуют во время выполнения или
исполнения программной системы, например, исполняемые компоненты. Те
элементы, которые не используются на этапе выполнения на диаграмме
развертывания, как правило, не показываются. Так, например, компоненты с
исходными текстами программ могут присутствовать на диаграмме
компонентов, а на диаграмме развертывания они не указываются, а может
указываться только исполняемый компонент, получаемый в результате их
компиляции.\par
В данной работе (рис. \ref{fig:img}) система продажи билетов состоит
из сервера сайта и системы оплаты.
Диаграмма развертывания используется менеджером проекта, пользователями,
архитектором системы и эксплуатационным персоналом, чтобы понять
физическое размещение системы и расположение ее отдельных подсистем.

\img{img}{Диаграмма компонентов}


\clearpage

\section*{Ответы на вопросы}
\addcontentsline{toc}{section}{Ответы на вопросы}

\begin{description}[style =sameline]
	\item [Какова роль диаграмм развертывания в проектировании
		информационных систем?]
		\textbf{Диаграмма развертывания (deployment diagram)} предназначена
		для представления общей конфигурации или топологии распределения
		программной системы и содержит изображение размещения различных
		артефактов (исполняемых компонентов и динамических библиотек) по
		отдельным узлам системы.
	\item [Какие группы специалистов участвуют в разработке диаграмм
		развертывания?]
		Обычно это могут быть:
		\begin{itemize}
			\item Архитекторы;
			\item Инженеры по развертыванию;
			\item Специалисты по безопасности;
			\item Специалисты по сопровождению и эксплуатации.
		\end{itemize}

		Диаграмма развертывания описывает расположение и взаимодействие
		компонентов системы на физических устройствах,
		поэтому она может описывать только такую ее часть,
		как конкретное размещение компонентов на серверах.
	\item [Является ли диаграмма развертывания единой для системы в целом или
		может ли она описывать какую-то ее часть?]
		Диаграмма развертывания проектируется, как правило,
		в единственном экземпляре, так как она должна всецело отражать
		особенности топологии и реализации разрабатываемой системы.
	\item [В каких случаях использование диаграммы развертывания
		нецелесообразно?]
		Использование диаграммы развертывания может быть нецелесообразным
		в следующих случаях:
		\begin{itemize}
			\item Если система состоит только из одного компонента,
				который может быть размещен на одном физическом сервере.
			\item Если система запущена в облаке или на платформе,
				где размещение и управление серверами осуществляется
				провайдером. В этом случае диаграмма развертывания может
				быть создана провайдером и не требоваться от разработчика
				системы.
			\item Если система не имеет физических компонентов,
				где каждый компонент запускается в своем контейнере в облаке
				или на виртуальной машине. В этом случае диаграмма
				развертывания может быть заменена диаграммами,
				описывающими различные микросервисы и их взаимодействие.
		\end{itemize}
	\item [Что собой в языке UML представляет узел?]
		Узел (node) является элементом модели, который представляет некоторый
		вычислительный ресурс для развертывания на нем различных артефактов.
	\item [Какие виды узлов используются в UML?]
		Узлы:
		\begin{itemize}
			\item Среда выполнения (execution environment);
			\item Устройство (device).
		\end{itemize}
	\item [В чем разница между элементами узел и «Экземпляр узла»?]
		В диаграмме развертывания «узел» представляет физический компонент,
		который содержит программное обеспечение и/или аппаратное обеспечение,
		выполняющие определенные функции.
		«Экземпляр узла» относится к конкретному экземпляру узла,
		который создается в процессе развертывания системы на определенном
		физическом компьютере или сервере. \par
		Таким образом, «узел» является общим понятием, которое относится
		к типу физического компонента, а «Экземпляр узла» является конкретным
		представлением этого компонента в системе с уникальными параметрами
		и настройками.
	\item [Для чего на диаграмме развертывания могут быт представлены
		компоненты? Какие виды компонентов для этого используются?]
		Диаграмма развертывания представляет архитектуру размещения
		компонентов на физических узлах, что позволяет оценить требования к
		инфраструктуре для работы системы. На диаграмме могут представляться
		различные виды компонентов, такие как серверы приложений, базы данных,
		веб-серверы, клиентские приложения и т.д.
	\item [Какую роль на диаграмме развертывания играют интерфейсы?]
		Интерфейсы на диаграмме развертывания отображаются в виде связей
		между компонентами и узлами, и показывают, как компоненты
		взаимодействуют друг с другом.
		Интерфейсы могут быть как внутренними, если они
		предназначены для общения внутри компонента, так и внешними, если они
		предназначены для общения между компонентами и узлами.
	\item [Для чего на диаграмме развертывания используются сообщения и
		зависимости?]
		Сообщения на диаграмме развертывания показывают,
		какие данные передаются между компонентами и узлами,
		и какие протоколы используются для этого.
		Зависимости выражают связь между компонентами и узлами, показывая,
		что один компонент зависит от другого и не может работать без него.
		Например, клиентское приложение может зависеть от сервера приложений
		или базы данных, и без них не может выполнять свои функции.
\end{description}

\clearpage

\section*{\LARGE Вывод}
\addcontentsline{toc}{section}{Вывод}
Мы создали диаграмму развертывания.\par
При этом перед разработкой диаграммы развертывания индентифицировать:
категории (типы) пользователей; аппаратные, технические и другие типы
устройств; виды и требуемую пропускную способность каналов связи.
Были рассмотрены варианты прокладки новой или модернизации
существующей корпоративной сети организации.\par
Также в целях наглядного представления распределенной информационной
системы на диаграмме отобразили компоненты, интерфейсы и
связи между ними.

